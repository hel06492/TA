\documentclass[laporan.tex]{subfiles}

\begin{document}

\chapter*{ABSTRAK}
\addcontentsline{toc}{chapter}{ABSTRAK}

Persaingan pasar untuk buah-buahan saat ini menjadi semakin ketat dengan masuknya impor buah-buahan dari luar negeri ke pasar lokal. Otomatisasi dalam proses produksi diharapkan dapat mempercepat proses produksi dan meningkatkan kualitas buah-buahan lokal sehingga mampu bersaing dengan buah-buahan impor. Teknologi \emph{computer vision} telah diterapkan di bidang produksi pangan untuk otomatisasi pemeriksaan kualitas bahan pangan. Proses ini didasarkan pada pengolahan dan analisis citra bahan pangan. Pada skripsi ini dibahas penggunaan analisis citra untuk mengukur banyaknya cacat pada permukaan kulit jeruk keprok \emph{Citrus sinensis} dengan klasifikasi \emph{pixel} atas citra jeruk.

\vskip 1ex
Kata kunci: \emph{computer vision}, \emph{nearest neighbor classification}, \emph{quality control}, pengendalian mutu bahan pangan, kualitas buah jeruk keprok\\
xi + 52 halaman; 23 gambar; 13 tabel\\
Daftar acuan: 11 (1973-2013)

\end{document}
