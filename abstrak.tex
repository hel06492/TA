\documentclass[laporan.tex]{subfiles}

\begin{document}

\chapter*{ABSTRAK}
\addcontentsline{toc}{chapter}{ABSTRAK}

Peningkatan persaingan pasar untuk buah-buahan saat ini menjadi semakin ketat dengan masuknya impor buah-buahan dari luar negeri ke pasar lokal. Tentunya diperlukan peningkatan dalam proses produksi buah-buahan dalam negeri agar dapat bersaing dengan produk impor. Otomatisasi dalam proses produksi diharapkan dapat mempercepat proses produksi dan meningkatkan kualitas buah-buahan lokal sehingga mampu bersaing dengan buah-buahan impor.

Teknologi computer vision telah diterapkan di bidang produksi pangan untuk membantu pemeriksaan kualitas bahan pangan. Proses ini didasarkan pada pengolahan dan analisis citra bahan pangan. Teknologi ini telah berhasil diterapkan pada analisis produk daging, ikan, keju serta roti dan pizza.

Pada skripsi ini dibahas penggunaan analisis citra untuk mengukur banyaknya cacat pada permukaan kulit jeruk keprok \emph{Citrus sinensis} dengan klasifikasi \emph{pixel} atas citra jeruk.

Kata kunci: \emph{computer vision}, \emph{quality control}, pengendalian mutu bahan pangan, kualitas buah jeruk keprok

\end{document}
