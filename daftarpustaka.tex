\documentclass[laporan.tex]{subfiles}

\begin{document}

\begin{thebibliography}{99}
\bibitem{basuki} Basuki, A. Palandi, Fatchurrochman, F. J. (2005). Pengolahan Citra Digital Menggunakan Visual Basic. Graha Ilmu: Yogyakarta.
\bibitem{brosnan} Tadhg Brosnan, Sun, Da-Wen. (2004). Improving quality inspection of food products by computer vision––a review. \emph{Journal of Food Engineering, 61}, 3-16.
\bibitem{derisma} Dila Deswari, Hendrick, Derisma. (2013). Identifikasi Kematangan Buah Tomat Menggunakan Metode Backpropagation (Thesis). Universitas Andalas: Padang.
\bibitem{gon} Rafael C. Gonzales, Woods, Richard E. (2002). Digital Image Processing. Prentice Hall: New Jersey.
\bibitem{knuth} Knuth, Donald. (1973). The Art Of Computer Programming Volume 3. Addison Wesley: Boston.
\bibitem{kulkarni} Kulkarni, Arum D. (2001). Computer Vision and Fuzzy-Neural Systems. Prentice Hall: New Jersey.
\bibitem{liu} Liu, Ming Hui. (2009). Navel orange blemish identification for quality grading system (Thesis). Massey University: Albany.
\bibitem{schalkoff} Schalkoff, Robert J. (1989). Digital Image Processing and Computer Vision. John Wiley \& Sons: New York.
\bibitem{sni} Standard Nasional Indonesia. (2009). SNI 3165:2009 Jeruk Keprok. BSN: Jakarta.
%\bibitem{prihartono} Prihartono, et. al. Identifikasi Iris Mata Menggunakan Alihragam Wavelet Haar.  \emph{TRANSMISI, 13}, 71-75.
%\bibitem{zhang} Lei Zhang, Bao, Paul. (2002). Edge detection by scale multiplication in wavelet domain. \emph{Pattern Recognition Letters, 23}, 1771-1784.
\end{thebibliography}

\end{document}
