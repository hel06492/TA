\documentclass[laporan.tex]{subfiles}

\begin{document}

\chapter{Kesimpulan dan Saran}

\section{Kesimpulan}

Metode klasifikasi pixel dapat mendeteksi letak cacat pada kulit jeruk dengan baik. Kelemahan metode ini adalah hasil deteksi tidak mencakup seluruh permukaan jeruk yang ditampilkan pada citra jeruk. Selain itu metode ini juga tidak dapat mendeteksi cacat pada daerah yang berkilap. Diperlukan pencahayaan yang baik untuk mendapatkan hasil deteksi yang optimum.

\section{Saran}

Setelah mengadakan penelitian ini, penulis menyampaikan beberapa saran untuk penelitian selanjutnya:

\begin{enumerate}
\item Penelitian lebih lanjut mengenai pengaruh pemrosesan awal citra dengan penyesuaian histogram untuk meningkatkan efektivitas deteksi dengan metode klasifikasi \emph{pixel} ini.
\item Pengolahan bagian tepi jeruk yang tidak dimasukkan ke dalam proses deteksi cacat dengan \emph{clustering} atau metode klasifikasi lainnya untuk mendeteksi daerah cacat tambahan pada bagian tepi.
\end{enumerate}

\end{document}
