\documentclass[laporan.tex]{subfiles}

\begin{document}

\chapter*{HALAMAN PENGESAHAN DEWAN PENGUJI}
\addcontentsline{toc}{chapter}{HALAMAN PENGESAHAN DEWAN PENGUJI}
%\begin{center}
%\bfseries KLASIFIKASI KUALITAS TEKSTUR KULIT JERUK KEPROK DENGAN PENDEKATAN \emph{DISCRETE WAVELET TRANSFORM}
%\bfseries Oleh:
%\end{center}

\begin{tabular}{l c l}
Nama Pelaksana & : & Helmi Fitria Nugroho \\
NIM & : & A11.2011.06492 \\
Program Studi & : & Teknik Informatika - S1 \\
Fakultas & : & Ilmu Komputer \\
Judul Tugas Akhir & : & Pengukuran Cacat Tekstur Pada Kulit Jeruk Keprok \\% dengan Klasifikasi \emph{Pixel}
& & dengan Klasifikasi \emph{Pixel} \\
\end{tabular}

\vskip 3ex

\begin{center}
Tugas akhir ini telah diujikan dan dipertahankan dihadapan Dewan Penguji pada sidang tugas akhir tanggal 5 Agustus 2015. Menurut pandangan kami, tugas akhir ini memadai dari segi kualitas maupun kuantitas untuk tujuan penganugrahan gelar Sarjana Komputer (S.Kom.)
\vskip 4.5ex
Semarang, 5 Agustus 2015\\
Dewan Penguji \\
\end{center}


\begin{center}
\begin{tabular}{c p{10em} c}
& & \\
& & \\
Ricardus Anggi P., MCS & & Setia Astuti, S.Si, M.Kom \\
\cline{1-1}   \cline{3-3}
Anggota & & Anggota \\
& & \\
& & \\
\end{tabular}
\end{center}
\vskip 4ex
\begin{table}[h]
\centering
\begin{tabular}{c}
Erna Zuni Astuti, M.Kom \\
\hline
Ketua
\end{tabular}
\end{table}
\end{document}
