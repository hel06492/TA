\documentclass[laporan.tex]{subfiles}

\begin{document}

\chapter*{HALAMAN PENGESAHAN DEWAN PENGUJI}
\addcontentsline{toc}{chapter}{HALAMAN PENGESAHAN DEWAN PENGUJI}
%\begin{center}
%\bfseries KLASIFIKASI KUALITAS TEKSTUR KULIT JERUK KEPROK DENGAN PENDEKATAN \emph{DISCRETE WAVELET TRANSFORM}
%\bfseries Oleh:
%\end{center}

\begin{tabular}{l c l}
Nama Pelaksana & : & Helmi Fitria Nugroho \\
NIM & : & A11.2011.06492 \\
Program Studi & : & Teknik Informatika - S1 \\
Fakultas & : & Ilmu Komputer \\
Judul Tugas Akhir & : & Pengukuran Cacat Tekstur Pada Kulit Jeruk Keprok \\% dengan Klasifikasi \emph{Pixel}
& & Dengan Klasifikasi \emph{Pixel} \\
\end{tabular}

\vskip 3ex

Tugas akhir ini telah diujikan dan dipertahankan dihadapan Dewan Penguji pada Sidang tugas akhir tanggal 5 Agustus 2015. Menurut pandangan kami, tugas akhir inimemadai dari segi kualitas maupun kuantitas untuk tujuan penganugrahan gelar Sarjana Komputer (S.Kom.)

\begin{tabular}{p{18em} l}
& Semarang, \\
& Dewan Penguji \\
\end{tabular}


\begin{center}
\begin{tabular}{c p{10em} c}
& & \\
& & \\
& & \\
Anggota & & Anggota \\
& & \\
& & \\
& & \\
& & \\
\end{tabular}
\end{center}
\vskip 4ex
\begin{table}[h]
\centering
\begin{tabular}{c}
Ketua
\end{tabular}
\end{table}
\end{document}
