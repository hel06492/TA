\documentclass[laporan.tex]{subfiles}

\begin{document}

\chapter*{UCAPAN TERIMA KASIH}
\addcontentsline{toc}{chapter}{UCAPAN TERIMA KASIH}

Dengan memanjatkan puji syukur kehadirat Allah, Tuhan Yang Maha Pengasih dan Maha Penyayang yang telah melimpahkan segala rahmat, hidayah dan inayah-Nya kepada penulis sehingga laporan tugas akhir dengan judul "PENGUKURAN CACAT TEKSTUR PADA KULIT JERUK KEPROK DENGAN KLASIFIKASI \emph{PIXEL}" dapat penulis selesaikan sesuai dengan rencana karena dukungan dari berbagai pihak yang tidak ternilai besarnya. Oleh karena itu penulis menyampaikan terimakasih \mbox{kepada}:
\begin{enumerate}
%\item Orang tua tercinta, yang tak kenal lelah terus membimbing, melindungi dan mendukung saya untuk mencapai cita-cita, yang selalu sabar menghadapi tingkah laku saya yang sering menyusahkan.
\item Dr.Ir. Edi Noersasongko,M.Kom, selaku Rektor Universitas Dian Nuswantoro Semarang.
\item Dr. Abdul Syukur, selaku Dekan Fasilkom.
\item Heru Agus Santoso Ph.D., selaku Ka.Progdi Teknik Informatika.
\item Elkaf Rahmawan P., M.Kom. selaku pembimbing tugas akhir yang memberikan ide penelitian, memberikan informasi referensi yang penulis butuhkan dan bimbingan yang berkaitan dengan penelitian penulis.
\item Dosen-dosen pengampu di Fakultas Ilmu Komputer Teknik Informatika Universitas Dian Nuswantoro Semarang yang telah memberikan ilmu dan pengalamannya masing-masing, sehingga penulis dapat mengimplementasikan ilmu yang telah disampaikan. Semoga Tuhan yang Maha Esa memberikan balasan yang lebih besar kepada beliau-beliau, dan pada akhirnya penulis berharap bahwa penulisan laporan tugas akhir ini dapat bermanfaat dan berguna sebagaimana fungsinya.
%\item Mas Edwin yang meminjamkan laptop untuk mengerjakan TA.
\item Kawan-kawan komunitas di dunia nyata dan dunia maya, yang telah memotivasi saya untuk lekas mandiri dan mengejar langkah mereka.
\end{enumerate}

\end{document}
