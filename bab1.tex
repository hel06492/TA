\documentclass[laporan.tex]{subfiles}

\begin{document}

\chapter{PENDAHULUAN}

\section{Latar Belakang}

Di tengah pasar bebas saat ini, produk pangan lokal menghadapi persaingan ketat dengan produk pangan impor. Salah satu produk yang terdampak adalah buah-buahan lokal yang terpinggirkan karena masyarakat semakin menggemari buah-buahan impor.\footnote{(http://technology-indonesia.com/home/201-buah-lokal-vs-buah-impor)}

Menurut laporan dari Gabungan Importir Hasil Bumi Indonesia, impor produk hortikultura (buah dan sayur) yang dilakukan oleh Indonesia terhitung besar. Saat ini 85\% dari seluruh produk Hortikultura yang beredar dan dinikmati oleh konsumen di Indonesia merupakan produk impor. Selain itu, jumlah impor produk tersebut selalu meningkat setiap tahunnya. Sebagai contoh, berdasarkan data dari Badan Pusat Statistik (BPS) impor buah Indonesia dari Cina (sebagai negara pemasok buah impor terbesar ke Indonesia sepanjang tahun 2011 dan periode Januari‐Februari 2012) mengalami kenaikan dari angka US\$46,7 juta pada bulan Desember 2011 menjadi US\$62,6 juta pada bulan Januari 2012 dan dari angka US\$30 juta pada bulan Februari menjadi US\$48,2 juta pada bulan Maret di tahun yang sama. Selain itu impor buah dari Thailand juga mengalami kenaikan dari angka US\$10,95 juta pada bulan Juni 2012 menjadi US\$35,07 juta pada bulan Juli 2012 dan mencapai angka US\$40,55 juta pada bulan Agustus 2012.\footnote{(http://cwts.ugm.ac.id/2013/04/politik-perdagangan-buah-impor-indonesia-tahun-2011-2012/)}

Oleh karena itu diperlukan peningkatan efisiensi proses produksi buah-buahan lokal agar dapat bersaing dengan buah-buahan impor di pasar dalam negeri. Banyak hal yang dapat diperbaiki dalam proses produksi antara lain dengan penerapan otomatisasi untuk meningkatkan efisiensi dan akurasi pengendalian mutu produk.

Salah satu produk buah-buahan lokal yang dapat dikembangkan adalah buah jeruk keprok (Citrus sinensis). SNI telah menetapkan standard kualitas jeruk keprok yang dituangkan dalam SNI 3165-2009. Standard ini mengklasifikasikan mutu jeruk keprok menjadi tiga kelas berdasarkan kriteria  bentuk, warna kulit, dan cacat pada kulit.

Teknologi komputer berkembang semakin pesat belakangan ini dan memberikan banyak pengaruh baik besar maupun kecil bagi kehidupan menusia. Salah satu cabang teknologi komputer yang berkembang adalah pengolahan dan analisis citra yang dikenal dengan istilah computer vision.

Computer vision atau visi komputer adalah konstruksi deskripsi yang eksplisit dan bermakna untuk objek dari citra objek. Teknologi ini bertujuan untuk menirukan efek penglihatan manusia dengan mempersepsikan dan memahami sebuah citra secara elektronis.
Bidang produksi pangan adalah salah satu sektor industri yang memperoleh manfaat dari teknologi visi komputer. Visi komputer telah berhasil diterapkan untuk pengukuran objektif berbagai produk pertanian (He, Yang, Xue, \& Geng, 1998; Li \& Wang, 1999) dan makanan olahan (Sun, 2000; Wang \& Sun, 2001). Industri pangan menempati peringkat sepuluh besar pengguna teknologi visi komputer (Gunasekaran, 1996).


\section{Rumusan Masalah}

Rumusan masalah yang muncul dari latar belakang yang telah disajikan seperti diatas adalah bagaimana cara mendeteksi cacat tekstur pada kulit jeruk berdasarkan klasifikasi \emph{pixel} citra objek jeruk.

\section{Batasan Masalah}

Dalam penelitian ini, supaya tidak menyimpang dari maksud dan tujuan dari penulisan tugas akhir ini, maka penulis akan berfokus pada deteksi cacat tekstur buah jeruk keprok dengan klasifikasi \emph{pixel}. Penelitian hanya membahas pengukuran cacat pada tekstur tanpa mengukur parameter visual lain, dengan demikian hasil yang diperoleh tidak langsung dapat digunakan untuk mengklasifikasikan kualitas buah jeruk secara keseluruhan.

\section{Tujuan Penelitian}

Tujuan dari penelitian ini adalah untuk mendeteksi adanya cacat pada tekstur kulit buah jeruk.

\section{Manfaat Penelitian}

Penelitian ini ditujukan untuk membantu proses pengendalian mutu buah jeruk keprok yang akan dijual di pasaran baik lokal maupun ekspor. Diharapkan dengan penelitian ini produsen maupun eksportir dapat meningkatkan efektifitas produksi dengan otomatisasi pemeriksaan kualitas buah jeruk dari deteksi keberadaan cacat pada kulit buah, sehingga kapasitas kemampuan penyortiran jeruk dapat ditingkatkan.

\end{document}
